%%%%%%%%%%%%%%%%%%%%%%%%%%%%%%%%%%%%%%%%%
% Programming/Coding Assignment
% LaTeX Template
%
% This template has been downloaded from:
% http://www.latextemplates.com
%
% Original author:
% Ted Pavlic (http://www.tedpavlic.com)
%
% Note:
% The \lipsum[#] commands throughout this template generate dummy text
% to fill the template out. These commands should all be removed when 
% writing assignment content.
%
% This template uses a Perl script as an example snippet of code, most other
% languages are also usable. Configure them in the "CODE INCLUSION 
% CONFIGURATION" section.
%
%%%%%%%%%%%%%%%%%%%%%%%%%%%%%%%%%%%%%%%%%

%----------------------------------------------------------------------------------------
%	PACKAGES AND OTHER DOCUMENT CONFIGURATIONS
%----------------------------------------------------------------------------------------

\documentclass[11pt]{article}

\usepackage{fancyhdr} % Required for custom headers
\usepackage{lastpage} % Required to determine the last page for the footer
\usepackage{extramarks} % Required for headers and footers
\usepackage[usenames,dvipsnames]{color} % Required for custom colors
\usepackage{graphicx} % Required to insert images
\usepackage{subcaption}
\usepackage{listings} % Required for insertion of code
\usepackage{courier} % Required for the courier font
\usepackage{amsmath}
\usepackage{framed}

% Margins
\topmargin=-0.45in
\evensidemargin=0in
\oddsidemargin=0in
\textwidth=6.5in
\textheight=9.0in
\headsep=0.25in

\linespread{1.1} % Line spacing

% Set up the header and footer
\pagestyle{fancy}
\lhead{\hmwkAuthorName} % Top left header
\chead{\hmwkClass\ (\hmwkClassTime): \hmwkTitle} % Top center head
%\rhead{\firstxmark} % Top right header
\lfoot{\lastxmark} % Bottom left footer
\cfoot{} % Bottom center footer
\rfoot{Page\ \thepage\ of\ \protect\pageref{LastPage}} % Bottom right footer
\renewcommand\headrulewidth{0.4pt} % Size of the header rule
\renewcommand\footrulewidth{0.4pt} % Size of the footer rule

\setlength\parindent{0pt} % Removes all indentation from paragraphs

%----------------------------------------------------------------------------------------
%	CODE INCLUSION CONFIGURATION
%----------------------------------------------------------------------------------------

\definecolor{mygreen}{rgb}{0,0.6,0}
\definecolor{mygray}{rgb}{0.5,0.5,0.5}
\definecolor{mymauve}{rgb}{0.58,0,0.82}

\lstset{ %
  backgroundcolor=\color{white},   % choose the background color
  basicstyle=\footnotesize,        % size of fonts used for the code
  breaklines=true,                 % automatic line breaking only at whitespace
  captionpos=b,                    % sets the caption-position to bottom
  commentstyle=\color{mygreen},    % comment style
  escapeinside={\%*}{*)},          % if you want to add LaTeX within your code
  keywordstyle=\color{blue},       % keyword style
  stringstyle=\color{mymauve},     % string literal style
}

%----------------------------------------------------------------------------------------
%	DOCUMENT STRUCTURE COMMANDS
%	Skip this unless you know what you're doing
%----------------------------------------------------------------------------------------

% Header and footer for when a page split occurs within a problem environment
\newcommand{\enterProblemHeader}[1]{
%\nobreak\extramarks{#1}{#1 continued on next page\ldots}\nobreak
%\nobreak\extramarks{#1 (continued)}{#1 continued on next page\ldots}\nobreak
}

% Header and footer for when a page split occurs between problem environments
\newcommand{\exitProblemHeader}[1]{
%\nobreak\extramarks{#1 (continued)}{#1 continued on next page\ldots}\nobreak
%\nobreak\extramarks{#1}{}\nobreak
}

\setcounter{secnumdepth}{0} % Removes default section numbers
\newcounter{homeworkProblemCounter} % Creates a counter to keep track of the number of problems
\setcounter{homeworkProblemCounter}{0}

\newcommand{\homeworkProblemName}{}
\newenvironment{homeworkProblem}[1][Problem \arabic{homeworkProblemCounter}]{ % Makes a new environment called homeworkProblem which takes 1 argument (custom name) but the default is "Problem #"
\stepcounter{homeworkProblemCounter} % Increase counter for number of problems
\renewcommand{\homeworkProblemName}{#1} % Assign \homeworkProblemName the name of the problem
\section{\homeworkProblemName} % Make a section in the document with the custom problem count
\enterProblemHeader{\homeworkProblemName} % Header and footer within the environment
}{
\exitProblemHeader{\homeworkProblemName} % Header and footer after the environment
}

\newcommand{\problemAnswer}[1]{ % Defines the problem answer command with the content as the only argument
\noindent\framebox[\columnwidth][c]{\begin{minipage}{0.98\columnwidth}#1\end{minipage}} % Makes the box around the problem answer and puts the content inside
}

\newcommand{\homeworkSectionName}{}
\newenvironment{homeworkSection}[1]{ % New environment for sections within homework problems, takes 1 argument - the name of the section
\renewcommand{\homeworkSectionName}{#1} % Assign \homeworkSectionName to the name of the section from the environment argument
\subsection{\homeworkSectionName} % Make a subsection with the custom name of the subsection
\enterProblemHeader{\homeworkProblemName\ [\homeworkSectionName]} % Header and footer within the environment
}{
\enterProblemHeader{\homeworkProblemName} % Header and footer after the environment
}

%----------------------------------------------------------------------------------------
%	NAME AND CLASS SECTION
%----------------------------------------------------------------------------------------

\newcommand{\hmwkTitle}{Problem Set 1} % Assignment title
\newcommand{\hmwkDueDate}{Friday, Oct 12, 2018} % Due date
\newcommand{\hmwkClass}{CSC336} % Course/class
\newcommand{\hmwkClassTime}{LEC 0101} % Class/lecture time
\newcommand{\hmwkAuthorName}{Zhongtian Ouyang} % Your name

%----------------------------------------------------------------------------------------
%	TITLE PAGE
%----------------------------------------------------------------------------------------

\title{
\vspace{2in}
\textmd{\textbf{\hmwkClass:\ \hmwkTitle}}\\
\normalsize\vspace{0.1in}\small{Due\ on\ \hmwkDueDate}\\
\vspace{0.1in}
\vspace{3in}
}

\author{\textbf{\hmwkAuthorName}}
\date{} % Insert date here if you want it to appear below your name

%----------------------------------------------------------------------------------------

\begin{document}

\maketitle
\clearpage
%----------------------------------------------------------------------------------------
%	PROBLEM 1
%----------------------------------------------------------------------------------------

% To have just one problem per page, simply put a \clearpage after each problem
%\begin{figure}[h!]
%\centering
%\includegraphics[width=0.6\linewidth]{q10a.png}
%\label{fig:q10a}
%\end{figure}\\

\begin{homeworkProblem}

\noindent \textit{Errors}\\

We assume $\pi$ equals to 3.14159265358979 in this question\\

a)\\
Absolute Error = $3.14 - \pi = -0.00159265358979 \approx -1.59 \times 10^{-3}$\\
Relative Error = $\frac{3.14 - \pi}{\pi} \approx  -5.07 \times 10^{-4}$\\

b)\\
Absolute Error = $3.14159 - \pi = -0.00000265358979 \approx -2.65 \times 10^{-3}$\\
Relative Error = $\frac{3.14159 - \pi}{\pi} \approx  -8.45 \times 10^{-7}$\\

c)\\
Absolute Error = $3.141592654 - \pi = 0.00000000041021 \approx 4.10 \times 10^{-10}$\\
Relative Error = $\frac{3.14 - \pi}{\pi} \approx  1.31 \times 10^{-10}$\\


\end{homeworkProblem}
%----------------------------------------------------------------------------------------
%	PROBLEM 2
%----------------------------------------------------------------------------------------

\begin{homeworkProblem}
\noindent \textit{Rounding}\\

(a)\\
$$5.35 \cdot 10^{0} + 2.46 \cdot 10^{-2}= 5.3746 \approx 5.37 \cdot 10^0$$
(b)\\
$$4.53 \cdot 10^{1} - 6.38 \cdot 10^{-1} = 44.662 \approx 4.47 \cdot 10^1$$
(c) \\
$$5.65 \cdot 10^{1 } + 5.23 \cdot 10^{-4}= 5.650523 \approx 5.65 \cdot 10^0$$
(d)\\
$$ 6.54 \cdot 10^{4} - 8.73 \cdot 10^{6} = -8.6646 \cdot 10^6 \approx -8.66 \cdot 10^6$$
(e)\\
$$ 5.21 \cdot 10^{8} \times 4.25 \cdot 10^{-5}= 22.1425  \cdot 10^3 = 2.21425 \cdot 10^4 \approx 2.21 \cdot 10^4$$
(f) \\
$$-4.32 \cdot 10^{7} \times 3.25 \cdot 10^{3}= -14.04 \cdot 10^{10}  = -1.404 \cdot 10^{11}\approx -inf$$
(g)\\
$$ 5.41 \cdot 10^{-5 }\times 4.27 \cdot 10^{-5}=23.1007 \cdot 10^{-10} = 2.31007 \cdot 10^{-9} \approx 2.31 \cdot 10^{-9}$$
(h)\\
$$ -6.52 \cdot 10^{-6 } \times 4.75 \cdot 10^{-6} = -30.97 \cdot 10^{-12} = -0.3097 \cdot 10^{-10} \approx -0.310 \cdot 10^{-10}$$
(i)\\
$$ -6.46 \cdot 10^{-7} \times 1.32 \cdot 10^{-6}= -8.5272 \cdot 10^{-13} = 0.0085272 \cdot 10^{-10} \approx 0.01 \cdot 10^{-10}$$
(j)\\
$$ 3.82 \cdot 10^{-6} \times 1.25 \cdot 10^{-7}= 4.775 \cdot 10^{-13} = 0.004775 \cdot 10^{-10} \approx 0.00 \cdot 10^{-10} = 0$$
\end{homeworkProblem}
%----------------------------------------------------------------------------------------
%	PROBLEM 3
%----------------------------------------------------------------------------------------

\begin{homeworkProblem}
\noindent \textit{Conditioning}\\

a)\\
$f(x) = tan(x) \to f'(x) = sec^2(x)$\\
The condtion number is:
$$\frac{xf'(x)}{f(x)} = \frac{x\cdot sec^2(x)}{tan(x)} =  \frac{x}{sin(x)cos(x)}$$
\\
When x is close to 0:\\
$$\lim_{x\to 0} \frac{xf'(x)}{f(x)} = \lim_{x\to 0}\frac{x}{sin(x)cos(x)}$$
Because both the numerator and denominator evaluates to zero, we can apply l'Hopital's Rule:
$$ \lim_{x\to 0}\frac{x}{sin(x)cos(x)} = \lim_{x\to 0} \frac{1}{cos^2(x) - sin^2(x)} =\lim_{x\to 0} \frac{1}{cos(2x)} = 1$$
Since 1 is means that we don't lose accuracy, this is well-conditioned.\\

When x is close to $\pi/2$:\\
$$\lim_{x\to \pi/2} \frac{xf'(x)}{f(x)} = \lim_{x\to \pi/2}\frac{x}{sin(x)cos(x)} = \infty$$
this is ill-conditioned.
\clearpage
b)\\
\begin{framed}
\begin{lstlisting}[language=matlab]
interval = 1e-8;
results1 = [];
for i = (0-2*interval):interval:(0+2*interval)
    results1 = [results1,tan(i)];
end
fprintf('%e ',results1);
fprintf('\n');

results2 = [];
val = pi/2;
for i = (val-5*interval):interval:(val-interval)
    results2 = [results2,tan(i)];
end
fprintf('%e ',results2);
fprintf('\n');
\end{lstlisting}
\end{framed}
The output from the program:
\begin{framed}
\begin{lstlisting}[language=matlab]
-2.000000e-08 -1.000000e-08 0.000000e+00 1.000000e-08 2.000000e-08 
2.000000e+07 2.500000e+07 3.333333e+07 5.000000e+07 1.000000e+08
\end{lstlisting}
\end{framed}
When x is close to 0, for example, suppose $x = 1.0 \cdot 10^{-8}$ and $\hat{x} = 2.0 \cdot 10^{-8}$, then $y = 1.0 \cdot 10^{-8}$ and $\hat{y} = 2.0 \cdot 10^{-8}$. The relative change in x = $\frac{\hat{x} - x}{x}$ = 1 and the relative change in y = $\frac{\hat{y} - y}{y}$ = 1. This support the conclusion that tan(x) is well-conditioned when x is close to 0.\\

When x is close to $\pi/2$, for example, suppose $x = \pi/2 - 2.0 \cdot 10^{-8}$ and $\hat{x} = \pi/2 - 1.0 \cdot 10^{-8}$, then $y = 5.0 \cdot 10^{7}$ and $\hat{y} = 1.0 \cdot 10^{8}$. The relative change in x = $\frac{\hat{x} - x}{x} \approx 6.366198 \cdot 10^{-9}$  and the relative change in y = $\frac{\hat{y} - y}{y}$ = 1. This support the conclusion that tan(x) is ill-conditioned when x is close to $\pi/2$.

\end{homeworkProblem}
\clearpage
%----------------------------------------------------------------------------------------
%	PROBLEM 4
%----------------------------------------------------------------------------------------

\begin{homeworkProblem}
\noindent \textit{2017 final}\\

Q2)\\
When x is a small but positive value, the true value of cos(x) is so close to 1 such that in Matlab, cos(x) is rounded to 1. so F evaluates to $(1-1) / x^2$, which is equals to zero. To fix this problem, we can rewrite the statement in the following way.	
$$F = \frac{(1-cos(x))}{x^2} = \frac{(1-cos(x))(1+cos(x))}{x^2 \cdot (1+cos(x))} = \frac{1-cos^2(x)}{x^2 \cdot (1+cos(x))} = \frac{sin^2(x)}{x^2 \cdot (1+cos(x))}$$\\

Q3)\\
Assume $|x| \geq |y|$,  the statement can be rewritten as:
$$G = \sqrt{x^2 + y^2} = |x| \cdot \frac{1}{|x|}\sqrt{x^2 + y^2} = |x|\sqrt{\frac{x^2}{|x|^2} + \frac{y^2}{|x|^2}} = |x|\sqrt{1 + (\frac{|y|}{|x|})^2} $$
The value of $ (\frac{|y|}{|x|})^2$ must between zero and one. So as long as x, y, and the final result can be expressed in IEEE Double-precision floating-point number, the intermediate values will not overflow or underflow.\\

if $|y| \geq |x|$, we can follow similar steps and get:
$$G = \sqrt{x^2 + y^2} = |y| \cdot \frac{1}{|y|}\sqrt{x^2 + y^2}= |y|\sqrt{(\frac{|x|}{|y|})^2 + 1 } $$
Below is an example code in Matlab using max and min function.
\begin{framed}
\begin{lstlisting}[language=matlab]
function result = f(x_in,y_in)
    x = abs(x_in);
    y = abs(y_in);
    if x == 0 && y == 0 
        result = 0;
    else
        nums = [x y];
        result = max(nums) * sqrt(1 + (min(nums)/max(nums))^2);
    end
end
\end{lstlisting}
\end{framed}

\end{homeworkProblem}
\clearpage
%----------------------------------------------------------------------------------------
%	PROBLEM 5
%----------------------------------------------------------------------------------------

\begin{homeworkProblem}
\noindent \textit{Matlab}\\
\\
The function exp1:
 \begin{framed}
\begin{lstlisting}[language=matlab]
function result = exp1(x)
    prev = -1;
    cur = 0;
    cycle = 0;
    
    while cur ~= prev
        prev = cur;
        cur = cur + (x^cycle)/factorial(cycle);
        cycle = cycle + 1;
    end
    
    result = cur;
end
\end{lstlisting}
\end{framed}
The program used to test x with value from -25 to 25:
 \begin{framed}
\begin{lstlisting}[language=matlab]
exp1s = [];
exps = [];
errors = [];
xs = [];

for x = -25:25
    exp1_result = exp1(x);
    exp_result = exp(x);
    relative_error = (exp1_result - exp_result)/exp_result;
    
    xs = [xs x];
    exp1s = [exp1s exp1_result];
    exps = [exps exp_result];
    errors = [errors relative_error];
end

A = [xs; exp1s; exps; errors];
fprintf('%6s %12s %12s %12s\n', 'x', 'exp1(x)', 'exp(x)', 'error');
fprintf('%6d %12.4e %12.4e %12.4e\n', A);
\end{lstlisting}
\end{framed}
\clearpage
The printed result from running the program:
\begin{framed}
\begin{lstlisting}[language=matlab]
     x      exp1(x)       exp(x)        error
   -25   8.0866e-07   1.3888e-11   5.8226e+04
   -24   3.7628e-07   3.7751e-11   9.9664e+03
   -23   6.8990e-09   1.0262e-10   6.6229e+01
   -22  -3.1820e-08   2.7895e-10  -1.1507e+02
   -21   2.7590e-08   7.5826e-10   3.5387e+01
   -20   4.1736e-09   2.0612e-09   1.0249e+00
   -19   2.5538e-09   5.6028e-09  -5.4420e-01
   -18   1.5984e-08   1.5230e-08   4.9480e-02
   -17   4.1442e-08   4.1399e-08   1.0196e-03
   -16   1.1257e-07   1.1254e-07   2.8526e-04
   -15   3.0591e-07   3.0590e-07   1.0354e-05
   -14   8.3152e-07   8.3153e-07  -8.6112e-06
   -13   2.2603e-06   2.2603e-06  -1.2997e-06
   -12   6.1442e-06   6.1442e-06   6.1218e-08
   -11   1.6702e-05   1.6702e-05   7.6483e-08
   -10   4.5400e-05   4.5400e-05  -7.2342e-09
    -9   1.2341e-04   1.2341e-04  -5.4918e-10
    -8   3.3546e-04   3.3546e-04  -1.4773e-10
    -7   9.1188e-04   9.1188e-04   1.2606e-11
    -6   2.4788e-03   2.4788e-03  -7.2503e-13
    -5   6.7379e-03   6.7379e-03   2.1369e-13
    -4   1.8316e-02   1.8316e-02   1.4396e-14
    -3   4.9787e-02   4.9787e-02   8.3623e-16
    -2   1.3534e-01   1.3534e-01   4.1018e-16
    -1   3.6788e-01   3.6788e-01   3.0179e-16
     0   1.0000e+00   1.0000e+00   0.0000e+00
     1   2.7183e+00   2.7183e+00   0.0000e+00
     2   7.3891e+00   7.3891e+00  -2.4040e-16
     3   2.0086e+01   2.0086e+01  -3.5376e-16
     4   5.4598e+01   5.4598e+01   5.2056e-16
     5   1.4841e+02   1.4841e+02  -1.9150e-16
     6   4.0343e+02   4.0343e+02   0.0000e+00
     7   1.0966e+03   1.0966e+03  -6.2201e-16
     8   2.9810e+03   2.9810e+03  -1.5255e-16
     9   8.1031e+03   8.1031e+03   0.0000e+00
    10   2.2026e+04   2.2026e+04  -3.3033e-16
    11   5.9874e+04   5.9874e+04   0.0000e+00
    12   1.6275e+05   1.6275e+05  -3.5764e-16
    13   4.4241e+05   4.4241e+05  -1.3157e-16
    14   1.2026e+06   1.2026e+06   1.9361e-16
    15   3.2690e+06   3.2690e+06   0.0000e+00
    16   8.8861e+06   8.8861e+06   0.0000e+00
    17   2.4155e+07   2.4155e+07   3.0845e-16
    18   6.5660e+07   6.5660e+07   0.0000e+00
    19   1.7848e+08   1.7848e+08  -1.6698e-16
    20   4.8517e+08   4.8517e+08  -2.4571e-16
    21   1.3188e+09   1.3188e+09  -3.6156e-16
    22   3.5849e+09   3.5849e+09   2.6602e-16
    23   9.7448e+09   9.7448e+09   1.9573e-16
    24   2.6489e+10   2.6489e+10   0.0000e+00
    25   7.2005e+10   7.2005e+10   0.0000e+00
\end{lstlisting}
\end{framed}
b)\\
When x is positive, the values are very accurate. But when x is negative, the approximations become more and more inaccurate as x gets smaller. The reason is that when x is positive, all terms are positive, we are accumulating to the final value; However, when x is negative, the even terms of the series is positive, while the odd terms of the series is negative. Therefore, the final result could be smaller than the intermidiate values. And when x gets smaller, the final result become closer to 0, while the intermediate values becomes larger. In lectures, we have shown that the if the intermediate values are much larger than the final result, the final result could be very inaccurate.\\

c)
\begin{framed}
\begin{lstlisting}[language=matlab]
function result = exp2(x_in)
    x = abs(x_in);
    prev = -1;
    cur = 0;
    cycle = 0;
    
    while cur ~= prev
        prev = cur;
        cur = cur + (x^cycle)/factorial(cycle);
        cycle = cycle + 1;
    end
    
    if x_in >= 0
        result = cur;
    else
        result = 1/cur;
    end
end
\end{lstlisting}
\end{framed}
The program for testing is almost identical to the one in part a), so I don't repeat it here.\\
We can see from the results below that all of the approximations are accurate now.
\begin{framed}
\begin{lstlisting}[language=matlab]
     x      exp2(x)       exp(x)        error
   -25   1.3888e-11   1.3888e-11  -1.1633e-16
   -24   3.7751e-11   3.7751e-11   0.0000e+00
   -23   1.0262e-10   1.0262e-10  -2.5190e-16
   -22   2.7895e-10   2.7895e-10  -1.8534e-16
   -21   7.5826e-10   7.5826e-10   4.0909e-16
   -20   2.0612e-09   2.0612e-09   2.0066e-16
   -19   5.6028e-09   5.6028e-09   0.0000e+00
   -18   1.5230e-08   1.5230e-08   0.0000e+00
   -17   4.1399e-08   4.1399e-08  -3.1969e-16
   -16   1.1254e-07   1.1254e-07   0.0000e+00
   -15   3.0590e-07   3.0590e-07   0.0000e+00
   -14   8.3153e-07   8.3153e-07  -2.5466e-16
   -13   2.2603e-06   2.2603e-06   1.8737e-16
   -12   6.1442e-06   6.1442e-06   4.1358e-16
   -11   1.6702e-05   1.6702e-05   0.0000e+00
   -10   4.5400e-05   4.5400e-05   2.9851e-16
    -9   1.2341e-04   1.2341e-04  -2.1963e-16
    -8   3.3546e-04   3.3546e-04   1.6160e-16
    -7   9.1188e-04   9.1188e-04   7.1338e-16
    -6   2.4788e-03   2.4788e-03   0.0000e+00
    -5   6.7379e-03   6.7379e-03   2.5746e-16
    -4   1.8316e-02   1.8316e-02  -3.7885e-16
    -3   4.9787e-02   4.9787e-02   2.7874e-16
    -2   1.3534e-01   1.3534e-01   2.0509e-16
    -1   3.6788e-01   3.6788e-01  -1.5089e-16
     0   1.0000e+00   1.0000e+00   0.0000e+00
     1   2.7183e+00   2.7183e+00   0.0000e+00
     2   7.3891e+00   7.3891e+00  -2.4040e-16
     3   2.0086e+01   2.0086e+01  -3.5376e-16
     4   5.4598e+01   5.4598e+01   5.2056e-16
     5   1.4841e+02   1.4841e+02  -1.9150e-16
     6   4.0343e+02   4.0343e+02   0.0000e+00
     7   1.0966e+03   1.0966e+03  -6.2201e-16
     8   2.9810e+03   2.9810e+03  -1.5255e-16
     9   8.1031e+03   8.1031e+03   0.0000e+00
    10   2.2026e+04   2.2026e+04  -3.3033e-16
    11   5.9874e+04   5.9874e+04   0.0000e+00
    12   1.6275e+05   1.6275e+05  -3.5764e-16
    13   4.4241e+05   4.4241e+05  -1.3157e-16
    14   1.2026e+06   1.2026e+06   1.9361e-16
    15   3.2690e+06   3.2690e+06   0.0000e+00
    16   8.8861e+06   8.8861e+06   0.0000e+00
    17   2.4155e+07   2.4155e+07   3.0845e-16
    18   6.5660e+07   6.5660e+07   0.0000e+00
    19   1.7848e+08   1.7848e+08  -1.6698e-16
    20   4.8517e+08   4.8517e+08  -2.4571e-16
    21   1.3188e+09   1.3188e+09  -3.6156e-16
    22   3.5849e+09   3.5849e+09   2.6602e-16
    23   9.7448e+09   9.7448e+09   1.9573e-16
    24   2.6489e+10   2.6489e+10   0.0000e+00
    25   7.2005e+10   7.2005e+10   0.0000e+00
\end{lstlisting}
\end{framed}
\end{homeworkProblem}
\clearpage

%----------------------------------------------------------------------------------------

\end{document}
